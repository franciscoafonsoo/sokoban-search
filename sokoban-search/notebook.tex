
% Default to the notebook output style

    


% Inherit from the specified cell style.




    
\documentclass[11pt]{article}

    
    
    \usepackage[T1]{fontenc}
    % Nicer default font (+ math font) than Computer Modern for most use cases
    \usepackage{mathpazo}

    % Basic figure setup, for now with no caption control since it's done
    % automatically by Pandoc (which extracts ![](path) syntax from Markdown).
    \usepackage{graphicx}
    % We will generate all images so they have a width \maxwidth. This means
    % that they will get their normal width if they fit onto the page, but
    % are scaled down if they would overflow the margins.
    \makeatletter
    \def\maxwidth{\ifdim\Gin@nat@width>\linewidth\linewidth
    \else\Gin@nat@width\fi}
    \makeatother
    \let\Oldincludegraphics\includegraphics
    % Set max figure width to be 80% of text width, for now hardcoded.
    \renewcommand{\includegraphics}[1]{\Oldincludegraphics[width=.8\maxwidth]{#1}}
    % Ensure that by default, figures have no caption (until we provide a
    % proper Figure object with a Caption API and a way to capture that
    % in the conversion process - todo).
    \usepackage{caption}
    \DeclareCaptionLabelFormat{nolabel}{}
    \captionsetup{labelformat=nolabel}

    \usepackage{adjustbox} % Used to constrain images to a maximum size 
    \usepackage{xcolor} % Allow colors to be defined
    \usepackage{enumerate} % Needed for markdown enumerations to work
    \usepackage{geometry} % Used to adjust the document margins
    \usepackage{amsmath} % Equations
    \usepackage{amssymb} % Equations
    \usepackage{textcomp} % defines textquotesingle
    % Hack from http://tex.stackexchange.com/a/47451/13684:
    \AtBeginDocument{%
        \def\PYZsq{\textquotesingle}% Upright quotes in Pygmentized code
    }
    \usepackage{upquote} % Upright quotes for verbatim code
    \usepackage{eurosym} % defines \euro
    \usepackage[mathletters]{ucs} % Extended unicode (utf-8) support
    \usepackage[utf8x]{inputenc} % Allow utf-8 characters in the tex document
    \usepackage{fancyvrb} % verbatim replacement that allows latex
    \usepackage{grffile} % extends the file name processing of package graphics 
                         % to support a larger range 
    % The hyperref package gives us a pdf with properly built
    % internal navigation ('pdf bookmarks' for the table of contents,
    % internal cross-reference links, web links for URLs, etc.)
    \usepackage{hyperref}
    \usepackage{longtable} % longtable support required by pandoc >1.10
    \usepackage{booktabs}  % table support for pandoc > 1.12.2
    \usepackage[inline]{enumitem} % IRkernel/repr support (it uses the enumerate* environment)
    \usepackage[normalem]{ulem} % ulem is needed to support strikethroughs (\sout)
                                % normalem makes italics be italics, not underlines
    

    
    
    % Colors for the hyperref package
    \definecolor{urlcolor}{rgb}{0,.145,.698}
    \definecolor{linkcolor}{rgb}{.71,0.21,0.01}
    \definecolor{citecolor}{rgb}{.12,.54,.11}

    % ANSI colors
    \definecolor{ansi-black}{HTML}{3E424D}
    \definecolor{ansi-black-intense}{HTML}{282C36}
    \definecolor{ansi-red}{HTML}{E75C58}
    \definecolor{ansi-red-intense}{HTML}{B22B31}
    \definecolor{ansi-green}{HTML}{00A250}
    \definecolor{ansi-green-intense}{HTML}{007427}
    \definecolor{ansi-yellow}{HTML}{DDB62B}
    \definecolor{ansi-yellow-intense}{HTML}{B27D12}
    \definecolor{ansi-blue}{HTML}{208FFB}
    \definecolor{ansi-blue-intense}{HTML}{0065CA}
    \definecolor{ansi-magenta}{HTML}{D160C4}
    \definecolor{ansi-magenta-intense}{HTML}{A03196}
    \definecolor{ansi-cyan}{HTML}{60C6C8}
    \definecolor{ansi-cyan-intense}{HTML}{258F8F}
    \definecolor{ansi-white}{HTML}{C5C1B4}
    \definecolor{ansi-white-intense}{HTML}{A1A6B2}

    % commands and environments needed by pandoc snippets
    % extracted from the output of `pandoc -s`
    \providecommand{\tightlist}{%
      \setlength{\itemsep}{0pt}\setlength{\parskip}{0pt}}
    \DefineVerbatimEnvironment{Highlighting}{Verbatim}{commandchars=\\\{\}}
    % Add ',fontsize=\small' for more characters per line
    \newenvironment{Shaded}{}{}
    \newcommand{\KeywordTok}[1]{\textcolor[rgb]{0.00,0.44,0.13}{\textbf{{#1}}}}
    \newcommand{\DataTypeTok}[1]{\textcolor[rgb]{0.56,0.13,0.00}{{#1}}}
    \newcommand{\DecValTok}[1]{\textcolor[rgb]{0.25,0.63,0.44}{{#1}}}
    \newcommand{\BaseNTok}[1]{\textcolor[rgb]{0.25,0.63,0.44}{{#1}}}
    \newcommand{\FloatTok}[1]{\textcolor[rgb]{0.25,0.63,0.44}{{#1}}}
    \newcommand{\CharTok}[1]{\textcolor[rgb]{0.25,0.44,0.63}{{#1}}}
    \newcommand{\StringTok}[1]{\textcolor[rgb]{0.25,0.44,0.63}{{#1}}}
    \newcommand{\CommentTok}[1]{\textcolor[rgb]{0.38,0.63,0.69}{\textit{{#1}}}}
    \newcommand{\OtherTok}[1]{\textcolor[rgb]{0.00,0.44,0.13}{{#1}}}
    \newcommand{\AlertTok}[1]{\textcolor[rgb]{1.00,0.00,0.00}{\textbf{{#1}}}}
    \newcommand{\FunctionTok}[1]{\textcolor[rgb]{0.02,0.16,0.49}{{#1}}}
    \newcommand{\RegionMarkerTok}[1]{{#1}}
    \newcommand{\ErrorTok}[1]{\textcolor[rgb]{1.00,0.00,0.00}{\textbf{{#1}}}}
    \newcommand{\NormalTok}[1]{{#1}}
    
    % Additional commands for more recent versions of Pandoc
    \newcommand{\ConstantTok}[1]{\textcolor[rgb]{0.53,0.00,0.00}{{#1}}}
    \newcommand{\SpecialCharTok}[1]{\textcolor[rgb]{0.25,0.44,0.63}{{#1}}}
    \newcommand{\VerbatimStringTok}[1]{\textcolor[rgb]{0.25,0.44,0.63}{{#1}}}
    \newcommand{\SpecialStringTok}[1]{\textcolor[rgb]{0.73,0.40,0.53}{{#1}}}
    \newcommand{\ImportTok}[1]{{#1}}
    \newcommand{\DocumentationTok}[1]{\textcolor[rgb]{0.73,0.13,0.13}{\textit{{#1}}}}
    \newcommand{\AnnotationTok}[1]{\textcolor[rgb]{0.38,0.63,0.69}{\textbf{\textit{{#1}}}}}
    \newcommand{\CommentVarTok}[1]{\textcolor[rgb]{0.38,0.63,0.69}{\textbf{\textit{{#1}}}}}
    \newcommand{\VariableTok}[1]{\textcolor[rgb]{0.10,0.09,0.49}{{#1}}}
    \newcommand{\ControlFlowTok}[1]{\textcolor[rgb]{0.00,0.44,0.13}{\textbf{{#1}}}}
    \newcommand{\OperatorTok}[1]{\textcolor[rgb]{0.40,0.40,0.40}{{#1}}}
    \newcommand{\BuiltInTok}[1]{{#1}}
    \newcommand{\ExtensionTok}[1]{{#1}}
    \newcommand{\PreprocessorTok}[1]{\textcolor[rgb]{0.74,0.48,0.00}{{#1}}}
    \newcommand{\AttributeTok}[1]{\textcolor[rgb]{0.49,0.56,0.16}{{#1}}}
    \newcommand{\InformationTok}[1]{\textcolor[rgb]{0.38,0.63,0.69}{\textbf{\textit{{#1}}}}}
    \newcommand{\WarningTok}[1]{\textcolor[rgb]{0.38,0.63,0.69}{\textbf{\textit{{#1}}}}}
    
    
    % Define a nice break command that doesn't care if a line doesn't already
    % exist.
    \def\br{\hspace*{\fill} \\* }
    % Math Jax compatability definitions
    \def\gt{>}
    \def\lt{<}
    % Document parameters
    \title{relatorio}
    
    
    

    % Pygments definitions
    
\makeatletter
\def\PY@reset{\let\PY@it=\relax \let\PY@bf=\relax%
    \let\PY@ul=\relax \let\PY@tc=\relax%
    \let\PY@bc=\relax \let\PY@ff=\relax}
\def\PY@tok#1{\csname PY@tok@#1\endcsname}
\def\PY@toks#1+{\ifx\relax#1\empty\else%
    \PY@tok{#1}\expandafter\PY@toks\fi}
\def\PY@do#1{\PY@bc{\PY@tc{\PY@ul{%
    \PY@it{\PY@bf{\PY@ff{#1}}}}}}}
\def\PY#1#2{\PY@reset\PY@toks#1+\relax+\PY@do{#2}}

\expandafter\def\csname PY@tok@w\endcsname{\def\PY@tc##1{\textcolor[rgb]{0.73,0.73,0.73}{##1}}}
\expandafter\def\csname PY@tok@c\endcsname{\let\PY@it=\textit\def\PY@tc##1{\textcolor[rgb]{0.25,0.50,0.50}{##1}}}
\expandafter\def\csname PY@tok@cp\endcsname{\def\PY@tc##1{\textcolor[rgb]{0.74,0.48,0.00}{##1}}}
\expandafter\def\csname PY@tok@k\endcsname{\let\PY@bf=\textbf\def\PY@tc##1{\textcolor[rgb]{0.00,0.50,0.00}{##1}}}
\expandafter\def\csname PY@tok@kp\endcsname{\def\PY@tc##1{\textcolor[rgb]{0.00,0.50,0.00}{##1}}}
\expandafter\def\csname PY@tok@kt\endcsname{\def\PY@tc##1{\textcolor[rgb]{0.69,0.00,0.25}{##1}}}
\expandafter\def\csname PY@tok@o\endcsname{\def\PY@tc##1{\textcolor[rgb]{0.40,0.40,0.40}{##1}}}
\expandafter\def\csname PY@tok@ow\endcsname{\let\PY@bf=\textbf\def\PY@tc##1{\textcolor[rgb]{0.67,0.13,1.00}{##1}}}
\expandafter\def\csname PY@tok@nb\endcsname{\def\PY@tc##1{\textcolor[rgb]{0.00,0.50,0.00}{##1}}}
\expandafter\def\csname PY@tok@nf\endcsname{\def\PY@tc##1{\textcolor[rgb]{0.00,0.00,1.00}{##1}}}
\expandafter\def\csname PY@tok@nc\endcsname{\let\PY@bf=\textbf\def\PY@tc##1{\textcolor[rgb]{0.00,0.00,1.00}{##1}}}
\expandafter\def\csname PY@tok@nn\endcsname{\let\PY@bf=\textbf\def\PY@tc##1{\textcolor[rgb]{0.00,0.00,1.00}{##1}}}
\expandafter\def\csname PY@tok@ne\endcsname{\let\PY@bf=\textbf\def\PY@tc##1{\textcolor[rgb]{0.82,0.25,0.23}{##1}}}
\expandafter\def\csname PY@tok@nv\endcsname{\def\PY@tc##1{\textcolor[rgb]{0.10,0.09,0.49}{##1}}}
\expandafter\def\csname PY@tok@no\endcsname{\def\PY@tc##1{\textcolor[rgb]{0.53,0.00,0.00}{##1}}}
\expandafter\def\csname PY@tok@nl\endcsname{\def\PY@tc##1{\textcolor[rgb]{0.63,0.63,0.00}{##1}}}
\expandafter\def\csname PY@tok@ni\endcsname{\let\PY@bf=\textbf\def\PY@tc##1{\textcolor[rgb]{0.60,0.60,0.60}{##1}}}
\expandafter\def\csname PY@tok@na\endcsname{\def\PY@tc##1{\textcolor[rgb]{0.49,0.56,0.16}{##1}}}
\expandafter\def\csname PY@tok@nt\endcsname{\let\PY@bf=\textbf\def\PY@tc##1{\textcolor[rgb]{0.00,0.50,0.00}{##1}}}
\expandafter\def\csname PY@tok@nd\endcsname{\def\PY@tc##1{\textcolor[rgb]{0.67,0.13,1.00}{##1}}}
\expandafter\def\csname PY@tok@s\endcsname{\def\PY@tc##1{\textcolor[rgb]{0.73,0.13,0.13}{##1}}}
\expandafter\def\csname PY@tok@sd\endcsname{\let\PY@it=\textit\def\PY@tc##1{\textcolor[rgb]{0.73,0.13,0.13}{##1}}}
\expandafter\def\csname PY@tok@si\endcsname{\let\PY@bf=\textbf\def\PY@tc##1{\textcolor[rgb]{0.73,0.40,0.53}{##1}}}
\expandafter\def\csname PY@tok@se\endcsname{\let\PY@bf=\textbf\def\PY@tc##1{\textcolor[rgb]{0.73,0.40,0.13}{##1}}}
\expandafter\def\csname PY@tok@sr\endcsname{\def\PY@tc##1{\textcolor[rgb]{0.73,0.40,0.53}{##1}}}
\expandafter\def\csname PY@tok@ss\endcsname{\def\PY@tc##1{\textcolor[rgb]{0.10,0.09,0.49}{##1}}}
\expandafter\def\csname PY@tok@sx\endcsname{\def\PY@tc##1{\textcolor[rgb]{0.00,0.50,0.00}{##1}}}
\expandafter\def\csname PY@tok@m\endcsname{\def\PY@tc##1{\textcolor[rgb]{0.40,0.40,0.40}{##1}}}
\expandafter\def\csname PY@tok@gh\endcsname{\let\PY@bf=\textbf\def\PY@tc##1{\textcolor[rgb]{0.00,0.00,0.50}{##1}}}
\expandafter\def\csname PY@tok@gu\endcsname{\let\PY@bf=\textbf\def\PY@tc##1{\textcolor[rgb]{0.50,0.00,0.50}{##1}}}
\expandafter\def\csname PY@tok@gd\endcsname{\def\PY@tc##1{\textcolor[rgb]{0.63,0.00,0.00}{##1}}}
\expandafter\def\csname PY@tok@gi\endcsname{\def\PY@tc##1{\textcolor[rgb]{0.00,0.63,0.00}{##1}}}
\expandafter\def\csname PY@tok@gr\endcsname{\def\PY@tc##1{\textcolor[rgb]{1.00,0.00,0.00}{##1}}}
\expandafter\def\csname PY@tok@ge\endcsname{\let\PY@it=\textit}
\expandafter\def\csname PY@tok@gs\endcsname{\let\PY@bf=\textbf}
\expandafter\def\csname PY@tok@gp\endcsname{\let\PY@bf=\textbf\def\PY@tc##1{\textcolor[rgb]{0.00,0.00,0.50}{##1}}}
\expandafter\def\csname PY@tok@go\endcsname{\def\PY@tc##1{\textcolor[rgb]{0.53,0.53,0.53}{##1}}}
\expandafter\def\csname PY@tok@gt\endcsname{\def\PY@tc##1{\textcolor[rgb]{0.00,0.27,0.87}{##1}}}
\expandafter\def\csname PY@tok@err\endcsname{\def\PY@bc##1{\setlength{\fboxsep}{0pt}\fcolorbox[rgb]{1.00,0.00,0.00}{1,1,1}{\strut ##1}}}
\expandafter\def\csname PY@tok@kc\endcsname{\let\PY@bf=\textbf\def\PY@tc##1{\textcolor[rgb]{0.00,0.50,0.00}{##1}}}
\expandafter\def\csname PY@tok@kd\endcsname{\let\PY@bf=\textbf\def\PY@tc##1{\textcolor[rgb]{0.00,0.50,0.00}{##1}}}
\expandafter\def\csname PY@tok@kn\endcsname{\let\PY@bf=\textbf\def\PY@tc##1{\textcolor[rgb]{0.00,0.50,0.00}{##1}}}
\expandafter\def\csname PY@tok@kr\endcsname{\let\PY@bf=\textbf\def\PY@tc##1{\textcolor[rgb]{0.00,0.50,0.00}{##1}}}
\expandafter\def\csname PY@tok@bp\endcsname{\def\PY@tc##1{\textcolor[rgb]{0.00,0.50,0.00}{##1}}}
\expandafter\def\csname PY@tok@fm\endcsname{\def\PY@tc##1{\textcolor[rgb]{0.00,0.00,1.00}{##1}}}
\expandafter\def\csname PY@tok@vc\endcsname{\def\PY@tc##1{\textcolor[rgb]{0.10,0.09,0.49}{##1}}}
\expandafter\def\csname PY@tok@vg\endcsname{\def\PY@tc##1{\textcolor[rgb]{0.10,0.09,0.49}{##1}}}
\expandafter\def\csname PY@tok@vi\endcsname{\def\PY@tc##1{\textcolor[rgb]{0.10,0.09,0.49}{##1}}}
\expandafter\def\csname PY@tok@vm\endcsname{\def\PY@tc##1{\textcolor[rgb]{0.10,0.09,0.49}{##1}}}
\expandafter\def\csname PY@tok@sa\endcsname{\def\PY@tc##1{\textcolor[rgb]{0.73,0.13,0.13}{##1}}}
\expandafter\def\csname PY@tok@sb\endcsname{\def\PY@tc##1{\textcolor[rgb]{0.73,0.13,0.13}{##1}}}
\expandafter\def\csname PY@tok@sc\endcsname{\def\PY@tc##1{\textcolor[rgb]{0.73,0.13,0.13}{##1}}}
\expandafter\def\csname PY@tok@dl\endcsname{\def\PY@tc##1{\textcolor[rgb]{0.73,0.13,0.13}{##1}}}
\expandafter\def\csname PY@tok@s2\endcsname{\def\PY@tc##1{\textcolor[rgb]{0.73,0.13,0.13}{##1}}}
\expandafter\def\csname PY@tok@sh\endcsname{\def\PY@tc##1{\textcolor[rgb]{0.73,0.13,0.13}{##1}}}
\expandafter\def\csname PY@tok@s1\endcsname{\def\PY@tc##1{\textcolor[rgb]{0.73,0.13,0.13}{##1}}}
\expandafter\def\csname PY@tok@mb\endcsname{\def\PY@tc##1{\textcolor[rgb]{0.40,0.40,0.40}{##1}}}
\expandafter\def\csname PY@tok@mf\endcsname{\def\PY@tc##1{\textcolor[rgb]{0.40,0.40,0.40}{##1}}}
\expandafter\def\csname PY@tok@mh\endcsname{\def\PY@tc##1{\textcolor[rgb]{0.40,0.40,0.40}{##1}}}
\expandafter\def\csname PY@tok@mi\endcsname{\def\PY@tc##1{\textcolor[rgb]{0.40,0.40,0.40}{##1}}}
\expandafter\def\csname PY@tok@il\endcsname{\def\PY@tc##1{\textcolor[rgb]{0.40,0.40,0.40}{##1}}}
\expandafter\def\csname PY@tok@mo\endcsname{\def\PY@tc##1{\textcolor[rgb]{0.40,0.40,0.40}{##1}}}
\expandafter\def\csname PY@tok@ch\endcsname{\let\PY@it=\textit\def\PY@tc##1{\textcolor[rgb]{0.25,0.50,0.50}{##1}}}
\expandafter\def\csname PY@tok@cm\endcsname{\let\PY@it=\textit\def\PY@tc##1{\textcolor[rgb]{0.25,0.50,0.50}{##1}}}
\expandafter\def\csname PY@tok@cpf\endcsname{\let\PY@it=\textit\def\PY@tc##1{\textcolor[rgb]{0.25,0.50,0.50}{##1}}}
\expandafter\def\csname PY@tok@c1\endcsname{\let\PY@it=\textit\def\PY@tc##1{\textcolor[rgb]{0.25,0.50,0.50}{##1}}}
\expandafter\def\csname PY@tok@cs\endcsname{\let\PY@it=\textit\def\PY@tc##1{\textcolor[rgb]{0.25,0.50,0.50}{##1}}}

\def\PYZbs{\char`\\}
\def\PYZus{\char`\_}
\def\PYZob{\char`\{}
\def\PYZcb{\char`\}}
\def\PYZca{\char`\^}
\def\PYZam{\char`\&}
\def\PYZlt{\char`\<}
\def\PYZgt{\char`\>}
\def\PYZsh{\char`\#}
\def\PYZpc{\char`\%}
\def\PYZdl{\char`\$}
\def\PYZhy{\char`\-}
\def\PYZsq{\char`\'}
\def\PYZdq{\char`\"}
\def\PYZti{\char`\~}
% for compatibility with earlier versions
\def\PYZat{@}
\def\PYZlb{[}
\def\PYZrb{]}
\makeatother


    % Exact colors from NB
    \definecolor{incolor}{rgb}{0.0, 0.0, 0.5}
    \definecolor{outcolor}{rgb}{0.545, 0.0, 0.0}



    
    % Prevent overflowing lines due to hard-to-break entities
    \sloppy 
    % Setup hyperref package
    \hypersetup{
      breaklinks=true,  % so long urls are correctly broken across lines
      colorlinks=true,
      urlcolor=urlcolor,
      linkcolor=linkcolor,
      citecolor=citecolor,
      }
    % Slightly bigger margins than the latex defaults
    
    \geometry{verbose,tmargin=1in,bmargin=1in,lmargin=1in,rmargin=1in}
    
    

    \begin{document}
    
    
    \maketitle
    
    

    
    \hypertarget{resoluuxe7uxe3o-de-puzzles-sokoban}{%
\section{Resolução de puzzles
Sokoban}\label{resoluuxe7uxe3o-de-puzzles-sokoban}}

    \hypertarget{formulauxe7uxe3o}{%
\subsection{Formulação}\label{formulauxe7uxe3o}}

\hypertarget{atributos}{%
\subsubsection{Atributos}\label{atributos}}

Foi escolhido representar um estado do puzzle sokoban usando uma class
\texttt{EstadoSokoban} que tem como atributos:

\begin{itemize}
\item
  \texttt{tabuleiro}: lista de listas que representa o puzzle: Cada uma
  das listas interiores vão ter:

  \begin{itemize}
  \item
    '\#' -- para representar as paredes;
  \item
    '.' -- para representar as posições livres;
  \item
    '*' -- para representar as caixas;
  \item
    'o' -- (um ó minúsculo) para representar os alvos;
  \item
    'A' -- para representar o arrumador.
  \item
    '@' -- para representar uma caixa numa posição alvo.
  \item
    'B' -- para representar o arrumador em cima de uma posição alvo.
  \end{itemize}
\item
  \texttt{arrumador}: posição do arrumador, tuplo xy.
\item
  \texttt{caixas}: posição das caixas, tuplo xy.
\item
  \texttt{alvos}: posição dos alvos, tuplo xy.
\item
  \texttt{deadlocks}: posição de deadlocks, executando
  \texttt{deadlocks\_tabuleiro()}
\end{itemize}

\hypertarget{muxe9todos}{%
\subsubsection{Métodos}\label{muxe9todos}}

Para além da representação, foram implementados os seguintes métodos:

\begin{itemize}
\item
  \texttt{pos\_livre(self,\ x,\ y)}: verifica se uma posição do puzzle
  está livre, isto é, se o arrumador por andar para lá.
\item
  \texttt{pos\_caixa(self,\ x,\ y)}: verifica se uma posição do puzzle
  tem uma caixa.
\item
  \texttt{ver\_*(self,\ x,\ y)}: usa as funções anteriores para
  verificar a posição ``cima'', ``baixo'', ``direita'' e ``esquerda''
  (cada uma destas opções são um método diferente.
\end{itemize}

\hypertarget{nota-sobre-este-relatuxf3rio}{%
\subsection{Nota sobre este
relatório}\label{nota-sobre-este-relatuxf3rio}}

Utilizamos a ferramenta Jupyter Notebook, para tirar proveito de algumas
funcionalidades do IPython para análise estatística da execução dos
algoritmos.

    \hypertarget{heuruxedsticas-definidas}{%
\subsection{Heurísticas definidas}\label{heuruxedsticas-definidas}}

    Considere-se \(n\) caixas e \(m\) alvos. No jogo sokoban, \(n=m\) para
qualquer jogo.

    Vamos então construir um grafo bipartido completo \(K_{n,m}\).

    Ao associar um custo às arestas de cada nodo caixa a um nodo alvo (neste
caso baseado na distância de manhattan), passamos a ter um problema de
afetação, para o qual existe uma série de algoritmos que, sabendo que
cada nodo n só vai estar asssociado a um nodo m, permitem escolher qual
a combinação de menor custo.

    Nesta heurística, foi escolhido o algoritmo hungáro (dado que é um
algoritmo estudado no âmbito das cadeira de Investigação Operacional e
Grafos e Redes) para associar uma caixa a um alvo, baseado na melhor
combinação de custos (Minimizar o custo global de \(K_{n,m}\)).

    A complexidade temporal deste algoritmo é de \(O(n^3)\), que é bastante
pesado computacional- mente, mas visto que nos puzzles dados n ≤ 3, vai
ser possivel calcular este algoritmo em tempo útil, como vamos testar na
proximá secção. Para não implementar o algoritmo hungáro de raíz, foi
utilizada a implementação {[}1{]}: https://github.com/bmc/munkres. Todo
o código necessário está disponivel no ficheiro hungarian.py

    \hypertarget{exemplos-de-execuuxe7uxe3o}{%
\subsection{Exemplos de execução}\label{exemplos-de-execuuxe7uxe3o}}

    No ficheiro \texttt{sokoban.py} estão as classes principais para a
execução do puzzle. Todo o código de análise execução (apresentado neste
relatório) está no ficheiro \texttt{run-sokoban.py}.

    \begin{Verbatim}[commandchars=\\\{\}]
{\color{incolor}In [{\color{incolor}1}]:} \PY{o}{\PYZpc{}}\PY{k}{run} \PYZhy{}t sokoban.py
\end{Verbatim}


    \begin{Verbatim}[commandchars=\\\{\}]

IPython CPU timings (estimated):
  User   :       0.01 s.
  System :       0.00 s.
Wall time:       0.01 s.

    \end{Verbatim}

    \hypertarget{anuxe1lise-dos-algoritmos-experimentados}{%
\subsection{Análise dos algoritmos
experimentados}\label{anuxe1lise-dos-algoritmos-experimentados}}

    Vai ser utilizado o puzzle2, entregue no enunciado, para testar os
algorimos disponíveis no ficheiro \texttt{search.py}, do repositório
aima-python, disponibilizado nas aulas.

    A função (definida em baixo) \texttt{statistics} é método para imprimir
dados da resolução de um problema Sokoban.

    \begin{Verbatim}[commandchars=\\\{\}]
{\color{incolor}In [{\color{incolor}11}]:} \PY{n}{sokoban} \PY{o}{=} \PY{n}{Sokoban}\PY{p}{(}\PY{n}{puzzle2}\PY{p}{)}
         
         \PY{k}{def} \PY{n+nf}{statistics}\PY{p}{(}\PY{n}{resultado}\PY{p}{,} \PY{n}{caminho}\PY{o}{=}\PY{k+kc}{False}\PY{p}{)}\PY{p}{:}
             \PY{n}{path} \PY{o}{=} \PY{n}{resultado}\PY{o}{.}\PY{n}{path}\PY{p}{(}\PY{p}{)}
             \PY{n}{solucao} \PY{o}{=} \PY{n}{resultado}\PY{o}{.}\PY{n}{solution}\PY{p}{(}\PY{p}{)}
             \PY{n}{number\PYZus{}moves} \PY{o}{=} \PY{l+m+mi}{0}
             \PY{n}{number\PYZus{}pushes} \PY{o}{=} \PY{l+m+mi}{0}
         
             \PY{k}{for} \PY{n}{index}\PY{p}{,} \PY{n}{action} \PY{o+ow}{in} \PY{n+nb}{enumerate}\PY{p}{(}\PY{n}{solucao}\PY{p}{)}\PY{p}{:}
                 \PY{n}{accao}\PY{p}{,} \PY{n}{\PYZus{}} \PY{o}{=} \PY{n}{action}\PY{o}{.}\PY{n}{split}\PY{p}{(}\PY{p}{)}
                 \PY{k}{if} \PY{n}{accao} \PY{o}{==} \PY{l+s+s1}{\PYZsq{}}\PY{l+s+s1}{andar}\PY{l+s+s1}{\PYZsq{}}\PY{p}{:}
                     \PY{n}{number\PYZus{}moves} \PY{o}{+}\PY{o}{=} \PY{l+m+mi}{1}
                 \PY{k}{else}\PY{p}{:}
                     \PY{n}{number\PYZus{}pushes} \PY{o}{+}\PY{o}{=} \PY{l+m+mi}{1}
         
             \PY{k}{for} \PY{n}{index}\PY{p}{,} \PY{n}{state} \PY{o+ow}{in} \PY{n+nb}{enumerate}\PY{p}{(}\PY{n}{path}\PY{p}{)}\PY{p}{:}
                 \PY{k}{if} \PY{n}{caminho}\PY{p}{:}
                     \PY{n+nb}{print}\PY{p}{(}\PY{n}{state}\PY{p}{)}
             \PY{k}{else}\PY{p}{:}
                 \PY{n+nb}{print}\PY{p}{(}\PY{l+s+s1}{\PYZsq{}}\PY{l+s+s1}{Número de passos:}\PY{l+s+s1}{\PYZsq{}}\PY{p}{,} \PY{n}{index}\PY{p}{)}
         
             \PY{n+nb}{print}\PY{p}{(}\PY{l+s+s1}{\PYZsq{}}\PY{l+s+s1}{Números de moves:}\PY{l+s+s1}{\PYZsq{}}\PY{p}{,} \PY{n}{number\PYZus{}moves}\PY{p}{)}
             \PY{n+nb}{print}\PY{p}{(}\PY{l+s+s1}{\PYZsq{}}\PY{l+s+s1}{Números de pushes:}\PY{l+s+s1}{\PYZsq{}}\PY{p}{,} \PY{n}{number\PYZus{}pushes}\PY{p}{)}
\end{Verbatim}


    \begin{Verbatim}[commandchars=\\\{\}]
{\color{incolor}In [{\color{incolor}3}]:} \PY{o}{\PYZpc{}\PYZpc{}}\PY{k}{timeit} 
        ucs\PYZus{}resultado = uniform\PYZus{}cost\PYZus{}search(sokoban)
\end{Verbatim}


    \begin{Verbatim}[commandchars=\\\{\}]
1.73 s ± 223 ms per loop (mean ± std. dev. of 7 runs, 1 loop each)

    \end{Verbatim}

    \begin{Verbatim}[commandchars=\\\{\}]
{\color{incolor}In [{\color{incolor}4}]:} \PY{o}{\PYZpc{}\PYZpc{}}\PY{k}{timeit}
        bfs\PYZus{}resultado = breadth\PYZus{}first\PYZus{}search(sokoban)
\end{Verbatim}


    \begin{Verbatim}[commandchars=\\\{\}]
1.7 s ± 254 ms per loop (mean ± std. dev. of 7 runs, 1 loop each)

    \end{Verbatim}

    \begin{Verbatim}[commandchars=\\\{\}]
{\color{incolor}In [{\color{incolor}5}]:} \PY{n}{bfs\PYZus{}resultado} \PY{o}{=} \PY{n}{breadth\PYZus{}first\PYZus{}search}\PY{p}{(}\PY{n}{sokoban}\PY{p}{)}
        \PY{n}{statistics}\PY{p}{(}\PY{n}{bfs\PYZus{}resultado}\PY{p}{)}
\end{Verbatim}


    \begin{Verbatim}[commandchars=\\\{\}]
Número de passos: 43
Números de moves: 31
Números de pushes: 12

    \end{Verbatim}

    \begin{Verbatim}[commandchars=\\\{\}]
{\color{incolor}In [{\color{incolor}6}]:} \PY{n}{ucs\PYZus{}resultado} \PY{o}{=} \PY{n}{uniform\PYZus{}cost\PYZus{}search}\PY{p}{(}\PY{n}{sokoban}\PY{p}{)}
        \PY{n}{statistics}\PY{p}{(}\PY{n}{ucs\PYZus{}resultado}\PY{p}{)}
\end{Verbatim}


    \begin{Verbatim}[commandchars=\\\{\}]
Número de passos: 43
Números de moves: 31
Números de pushes: 12

    \end{Verbatim}

    \begin{Verbatim}[commandchars=\\\{\}]
{\color{incolor}In [{\color{incolor}7}]:} \PY{o}{\PYZpc{}\PYZpc{}}\PY{k}{timeit}
        astar\PYZus{}resultado = astar\PYZus{}search(sokoban, hung\PYZus{}alg\PYZus{}manh)
\end{Verbatim}


    \begin{Verbatim}[commandchars=\\\{\}]
1.48 s ± 49.2 ms per loop (mean ± std. dev. of 7 runs, 1 loop each)

    \end{Verbatim}

    \begin{Verbatim}[commandchars=\\\{\}]
{\color{incolor}In [{\color{incolor}8}]:} \PY{k+kn}{from} \PY{n+nn}{hungarian} \PY{k}{import} \PY{n}{Munkres}
        
        \PY{k}{def} \PY{n+nf}{hung\PYZus{}alg\PYZus{}manh}\PY{p}{(}\PY{n}{nodo}\PY{p}{)}\PY{p}{:}
            \PY{l+s+sd}{\PYZdq{}\PYZdq{}\PYZdq{}Algoritmo hungaro, em que o custo de cada caixa a um alvo é a distância de manhattan.\PYZdq{}\PYZdq{}\PYZdq{}}
            \PY{n}{m} \PY{o}{=} \PY{n}{Munkres}\PY{p}{(}\PY{p}{)}
                
            \PY{n}{caixas} \PY{o}{=} \PY{n}{nodo}\PY{o}{.}\PY{n}{state}\PY{o}{.}\PY{n}{caixas}
            \PY{n}{alvos} \PY{o}{=} \PY{n}{nodo}\PY{o}{.}\PY{n}{state}\PY{o}{.}\PY{n}{alvos}
            \PY{n}{custo} \PY{o}{=} \PY{n+nb}{list}\PY{p}{(}\PY{p}{)}
            \PY{n}{mhd} \PY{o}{=} \PY{l+m+mi}{0}
        
            \PY{k}{for} \PY{n}{index}\PY{p}{,} \PY{n}{c} \PY{o+ow}{in} \PY{n+nb}{enumerate}\PY{p}{(}\PY{n}{caixas}\PY{p}{)}\PY{p}{:}
                \PY{n}{custo}\PY{o}{.}\PY{n}{append}\PY{p}{(}\PY{n+nb}{list}\PY{p}{(}\PY{p}{)}\PY{p}{)}
                \PY{k}{for} \PY{n}{a} \PY{o+ow}{in} \PY{n}{alvos}\PY{p}{:}
                    \PY{n}{custo}\PY{p}{[}\PY{n}{index}\PY{p}{]}\PY{o}{.}\PY{n}{append}\PY{p}{(}\PY{n+nb}{abs}\PY{p}{(}\PY{n}{c}\PY{p}{[}\PY{l+m+mi}{0}\PY{p}{]} \PY{o}{\PYZhy{}} \PY{n}{a}\PY{p}{[}\PY{l+m+mi}{0}\PY{p}{]}\PY{p}{)} \PY{o}{+} \PY{n+nb}{abs}\PY{p}{(}\PY{n}{c}\PY{p}{[}\PY{l+m+mi}{1}\PY{p}{]} \PY{o}{\PYZhy{}} \PY{n}{a}\PY{p}{[}\PY{l+m+mi}{1}\PY{p}{]}\PY{p}{)}\PY{p}{)}
            
            \PY{n}{indexes} \PY{o}{=} \PY{n}{m}\PY{o}{.}\PY{n}{compute}\PY{p}{(}\PY{n}{custo}\PY{p}{)}
            \PY{k}{for} \PY{n}{row}\PY{p}{,} \PY{n}{column} \PY{o+ow}{in} \PY{n}{indexes}\PY{p}{:}
                \PY{n}{value} \PY{o}{=} \PY{n}{custo}\PY{p}{[}\PY{n}{row}\PY{p}{]}\PY{p}{[}\PY{n}{column}\PY{p}{]}
                \PY{n}{mhd} \PY{o}{+}\PY{o}{=} \PY{n}{value}
            
            \PY{k}{return} \PY{n}{mhd}
\end{Verbatim}


    \begin{Verbatim}[commandchars=\\\{\}]
{\color{incolor}In [{\color{incolor}9}]:} \PY{n}{astar\PYZus{}resultado} \PY{o}{=} \PY{n}{astar\PYZus{}search}\PY{p}{(}\PY{n}{sokoban}\PY{p}{,} \PY{n}{hung\PYZus{}alg\PYZus{}manh}\PY{p}{)}
        \PY{n}{statistics}\PY{p}{(}\PY{n}{astar\PYZus{}resultado}\PY{p}{,} \PY{n}{caminho}\PY{o}{=}\PY{k+kc}{True}\PY{p}{)}
\end{Verbatim}


    \begin{Verbatim}[commandchars=\\\{\}]
\textbf{\#}\textbf{\#}\textbf{\#}\textbf{\#}\textbf{\#}\textbf{\#}\textbf{\#}
\textbf{\#}\textbf{\#}    \textbf{\#}
\textbf{\#}\textbf{\#}\textcolor{ansi-red-intense}{*} \textcolor{ansi-magenta-intense}{o} \textbf{\#}
\textbf{\#}  \textcolor{ansi-cyan}{A}  \textbf{\#}
\textbf{\#} \textcolor{ansi-red-intense}{*}\textbf{\#}\textcolor{ansi-magenta-intense}{o} \textbf{\#}
\textbf{\#}  \textbf{\#}\textbf{\#}\textbf{\#}\textbf{\#}
\textbf{\#}\textbf{\#}\textbf{\#}\textbf{\#}\textbf{\#}\textbf{\#}\textbf{\#}

\textbf{\#}\textbf{\#}\textbf{\#}\textbf{\#}\textbf{\#}\textbf{\#}\textbf{\#}
\textbf{\#}\textbf{\#}    \textbf{\#}
\textbf{\#}\textbf{\#}\textcolor{ansi-red-intense}{*} \textcolor{ansi-magenta-intense}{o} \textbf{\#}
\textbf{\#} \textcolor{ansi-cyan}{A}   \textbf{\#}
\textbf{\#} \textcolor{ansi-red-intense}{*}\textbf{\#}\textcolor{ansi-magenta-intense}{o} \textbf{\#}
\textbf{\#}  \textbf{\#}\textbf{\#}\textbf{\#}\textbf{\#}
\textbf{\#}\textbf{\#}\textbf{\#}\textbf{\#}\textbf{\#}\textbf{\#}\textbf{\#}

\textbf{\#}\textbf{\#}\textbf{\#}\textbf{\#}\textbf{\#}\textbf{\#}\textbf{\#}
\textbf{\#}\textbf{\#}    \textbf{\#}
\textbf{\#}\textbf{\#}\textcolor{ansi-red-intense}{*} \textcolor{ansi-magenta-intense}{o} \textbf{\#}
\textbf{\#}\textcolor{ansi-cyan}{A}    \textbf{\#}
\textbf{\#} \textcolor{ansi-red-intense}{*}\textbf{\#}\textcolor{ansi-magenta-intense}{o} \textbf{\#}
\textbf{\#}  \textbf{\#}\textbf{\#}\textbf{\#}\textbf{\#}
\textbf{\#}\textbf{\#}\textbf{\#}\textbf{\#}\textbf{\#}\textbf{\#}\textbf{\#}

\textbf{\#}\textbf{\#}\textbf{\#}\textbf{\#}\textbf{\#}\textbf{\#}\textbf{\#}
\textbf{\#}\textbf{\#}    \textbf{\#}
\textbf{\#}\textbf{\#}\textcolor{ansi-red-intense}{*} \textcolor{ansi-magenta-intense}{o} \textbf{\#}
\textbf{\#}     \textbf{\#}
\textbf{\#}\textcolor{ansi-cyan}{A}\textcolor{ansi-red-intense}{*}\textbf{\#}\textcolor{ansi-magenta-intense}{o} \textbf{\#}
\textbf{\#}  \textbf{\#}\textbf{\#}\textbf{\#}\textbf{\#}
\textbf{\#}\textbf{\#}\textbf{\#}\textbf{\#}\textbf{\#}\textbf{\#}\textbf{\#}

\textbf{\#}\textbf{\#}\textbf{\#}\textbf{\#}\textbf{\#}\textbf{\#}\textbf{\#}
\textbf{\#}\textbf{\#}    \textbf{\#}
\textbf{\#}\textbf{\#}\textcolor{ansi-red-intense}{*} \textcolor{ansi-magenta-intense}{o} \textbf{\#}
\textbf{\#}     \textbf{\#}
\textbf{\#} \textcolor{ansi-red-intense}{*}\textbf{\#}\textcolor{ansi-magenta-intense}{o} \textbf{\#}
\textbf{\#}\textcolor{ansi-cyan}{A} \textbf{\#}\textbf{\#}\textbf{\#}\textbf{\#}
\textbf{\#}\textbf{\#}\textbf{\#}\textbf{\#}\textbf{\#}\textbf{\#}\textbf{\#}

\textbf{\#}\textbf{\#}\textbf{\#}\textbf{\#}\textbf{\#}\textbf{\#}\textbf{\#}
\textbf{\#}\textbf{\#}    \textbf{\#}
\textbf{\#}\textbf{\#}\textcolor{ansi-red-intense}{*} \textcolor{ansi-magenta-intense}{o} \textbf{\#}
\textbf{\#}     \textbf{\#}
\textbf{\#} \textcolor{ansi-red-intense}{*}\textbf{\#}\textcolor{ansi-magenta-intense}{o} \textbf{\#}
\textbf{\#} \textcolor{ansi-cyan}{A}\textbf{\#}\textbf{\#}\textbf{\#}\textbf{\#}
\textbf{\#}\textbf{\#}\textbf{\#}\textbf{\#}\textbf{\#}\textbf{\#}\textbf{\#}

\textbf{\#}\textbf{\#}\textbf{\#}\textbf{\#}\textbf{\#}\textbf{\#}\textbf{\#}
\textbf{\#}\textbf{\#}    \textbf{\#}
\textbf{\#}\textbf{\#}\textcolor{ansi-red-intense}{*} \textcolor{ansi-magenta-intense}{o} \textbf{\#}
\textbf{\#} \textcolor{ansi-red-intense}{*}   \textbf{\#}
\textbf{\#} \textcolor{ansi-cyan}{A}\textbf{\#}\textcolor{ansi-magenta-intense}{o} \textbf{\#}
\textbf{\#}  \textbf{\#}\textbf{\#}\textbf{\#}\textbf{\#}
\textbf{\#}\textbf{\#}\textbf{\#}\textbf{\#}\textbf{\#}\textbf{\#}\textbf{\#}

\textbf{\#}\textbf{\#}\textbf{\#}\textbf{\#}\textbf{\#}\textbf{\#}\textbf{\#}
\textbf{\#}\textbf{\#}    \textbf{\#}
\textbf{\#}\textbf{\#}\textcolor{ansi-red-intense}{*} \textcolor{ansi-magenta-intense}{o} \textbf{\#}
\textbf{\#} \textcolor{ansi-red-intense}{*}   \textbf{\#}
\textbf{\#}\textcolor{ansi-cyan}{A} \textbf{\#}\textcolor{ansi-magenta-intense}{o} \textbf{\#}
\textbf{\#}  \textbf{\#}\textbf{\#}\textbf{\#}\textbf{\#}
\textbf{\#}\textbf{\#}\textbf{\#}\textbf{\#}\textbf{\#}\textbf{\#}\textbf{\#}

\textbf{\#}\textbf{\#}\textbf{\#}\textbf{\#}\textbf{\#}\textbf{\#}\textbf{\#}
\textbf{\#}\textbf{\#}    \textbf{\#}
\textbf{\#}\textbf{\#}\textcolor{ansi-red-intense}{*} \textcolor{ansi-magenta-intense}{o} \textbf{\#}
\textbf{\#}\textcolor{ansi-cyan}{A}\textcolor{ansi-red-intense}{*}   \textbf{\#}
\textbf{\#}  \textbf{\#}\textcolor{ansi-magenta-intense}{o} \textbf{\#}
\textbf{\#}  \textbf{\#}\textbf{\#}\textbf{\#}\textbf{\#}
\textbf{\#}\textbf{\#}\textbf{\#}\textbf{\#}\textbf{\#}\textbf{\#}\textbf{\#}

\textbf{\#}\textbf{\#}\textbf{\#}\textbf{\#}\textbf{\#}\textbf{\#}\textbf{\#}
\textbf{\#}\textbf{\#}    \textbf{\#}
\textbf{\#}\textbf{\#}\textcolor{ansi-red-intense}{*} \textcolor{ansi-magenta-intense}{o} \textbf{\#}
\textbf{\#} \textcolor{ansi-cyan}{A}\textcolor{ansi-red-intense}{*}  \textbf{\#}
\textbf{\#}  \textbf{\#}\textcolor{ansi-magenta-intense}{o} \textbf{\#}
\textbf{\#}  \textbf{\#}\textbf{\#}\textbf{\#}\textbf{\#}
\textbf{\#}\textbf{\#}\textbf{\#}\textbf{\#}\textbf{\#}\textbf{\#}\textbf{\#}

\textbf{\#}\textbf{\#}\textbf{\#}\textbf{\#}\textbf{\#}\textbf{\#}\textbf{\#}
\textbf{\#}\textbf{\#}    \textbf{\#}
\textbf{\#}\textbf{\#}\textcolor{ansi-red-intense}{*} \textcolor{ansi-magenta-intense}{o} \textbf{\#}
\textbf{\#}  \textcolor{ansi-cyan}{A}\textcolor{ansi-red-intense}{*} \textbf{\#}
\textbf{\#}  \textbf{\#}\textcolor{ansi-magenta-intense}{o} \textbf{\#}
\textbf{\#}  \textbf{\#}\textbf{\#}\textbf{\#}\textbf{\#}
\textbf{\#}\textbf{\#}\textbf{\#}\textbf{\#}\textbf{\#}\textbf{\#}\textbf{\#}

\textbf{\#}\textbf{\#}\textbf{\#}\textbf{\#}\textbf{\#}\textbf{\#}\textbf{\#}
\textbf{\#}\textbf{\#}    \textbf{\#}
\textbf{\#}\textbf{\#}\textcolor{ansi-red-intense}{*}\textcolor{ansi-cyan}{A}\textcolor{ansi-magenta-intense}{o} \textbf{\#}
\textbf{\#}   \textcolor{ansi-red-intense}{*} \textbf{\#}
\textbf{\#}  \textbf{\#}\textcolor{ansi-magenta-intense}{o} \textbf{\#}
\textbf{\#}  \textbf{\#}\textbf{\#}\textbf{\#}\textbf{\#}
\textbf{\#}\textbf{\#}\textbf{\#}\textbf{\#}\textbf{\#}\textbf{\#}\textbf{\#}

\textbf{\#}\textbf{\#}\textbf{\#}\textbf{\#}\textbf{\#}\textbf{\#}\textbf{\#}
\textbf{\#}\textbf{\#}    \textbf{\#}
\textbf{\#}\textbf{\#}\textcolor{ansi-red-intense}{*} \textcolor{ansi-cyan-intense}{B} \textbf{\#}
\textbf{\#}   \textcolor{ansi-red-intense}{*} \textbf{\#}
\textbf{\#}  \textbf{\#}\textcolor{ansi-magenta-intense}{o} \textbf{\#}
\textbf{\#}  \textbf{\#}\textbf{\#}\textbf{\#}\textbf{\#}
\textbf{\#}\textbf{\#}\textbf{\#}\textbf{\#}\textbf{\#}\textbf{\#}\textbf{\#}

\textbf{\#}\textbf{\#}\textbf{\#}\textbf{\#}\textbf{\#}\textbf{\#}\textbf{\#}
\textbf{\#}\textbf{\#}    \textbf{\#}
\textbf{\#}\textbf{\#}\textcolor{ansi-red-intense}{*} \textcolor{ansi-magenta-intense}{o}\textcolor{ansi-cyan}{A}\textbf{\#}
\textbf{\#}   \textcolor{ansi-red-intense}{*} \textbf{\#}
\textbf{\#}  \textbf{\#}\textcolor{ansi-magenta-intense}{o} \textbf{\#}
\textbf{\#}  \textbf{\#}\textbf{\#}\textbf{\#}\textbf{\#}
\textbf{\#}\textbf{\#}\textbf{\#}\textbf{\#}\textbf{\#}\textbf{\#}\textbf{\#}

\textbf{\#}\textbf{\#}\textbf{\#}\textbf{\#}\textbf{\#}\textbf{\#}\textbf{\#}
\textbf{\#}\textbf{\#}    \textbf{\#}
\textbf{\#}\textbf{\#}\textcolor{ansi-red-intense}{*} \textcolor{ansi-magenta-intense}{o} \textbf{\#}
\textbf{\#}   \textcolor{ansi-red-intense}{*}\textcolor{ansi-cyan}{A}\textbf{\#}
\textbf{\#}  \textbf{\#}\textcolor{ansi-magenta-intense}{o} \textbf{\#}
\textbf{\#}  \textbf{\#}\textbf{\#}\textbf{\#}\textbf{\#}
\textbf{\#}\textbf{\#}\textbf{\#}\textbf{\#}\textbf{\#}\textbf{\#}\textbf{\#}

\textbf{\#}\textbf{\#}\textbf{\#}\textbf{\#}\textbf{\#}\textbf{\#}\textbf{\#}
\textbf{\#}\textbf{\#}    \textbf{\#}
\textbf{\#}\textbf{\#}\textcolor{ansi-red-intense}{*} \textcolor{ansi-magenta-intense}{o} \textbf{\#}
\textbf{\#}   \textcolor{ansi-red-intense}{*} \textbf{\#}
\textbf{\#}  \textbf{\#}\textcolor{ansi-magenta-intense}{o}\textcolor{ansi-cyan}{A}\textbf{\#}
\textbf{\#}  \textbf{\#}\textbf{\#}\textbf{\#}\textbf{\#}
\textbf{\#}\textbf{\#}\textbf{\#}\textbf{\#}\textbf{\#}\textbf{\#}\textbf{\#}

\textbf{\#}\textbf{\#}\textbf{\#}\textbf{\#}\textbf{\#}\textbf{\#}\textbf{\#}
\textbf{\#}\textbf{\#}    \textbf{\#}
\textbf{\#}\textbf{\#}\textcolor{ansi-red-intense}{*} \textcolor{ansi-magenta-intense}{o} \textbf{\#}
\textbf{\#}   \textcolor{ansi-red-intense}{*} \textbf{\#}
\textbf{\#}  \textbf{\#}\textcolor{ansi-cyan-intense}{B} \textbf{\#}
\textbf{\#}  \textbf{\#}\textbf{\#}\textbf{\#}\textbf{\#}
\textbf{\#}\textbf{\#}\textbf{\#}\textbf{\#}\textbf{\#}\textbf{\#}\textbf{\#}

\textbf{\#}\textbf{\#}\textbf{\#}\textbf{\#}\textbf{\#}\textbf{\#}\textbf{\#}
\textbf{\#}\textbf{\#}    \textbf{\#}
\textbf{\#}\textbf{\#}\textcolor{ansi-red-intense}{*} \textcolor{ansi-blue-intense}{@} \textbf{\#}
\textbf{\#}   \textcolor{ansi-cyan}{A} \textbf{\#}
\textbf{\#}  \textbf{\#}\textcolor{ansi-magenta-intense}{o} \textbf{\#}
\textbf{\#}  \textbf{\#}\textbf{\#}\textbf{\#}\textbf{\#}
\textbf{\#}\textbf{\#}\textbf{\#}\textbf{\#}\textbf{\#}\textbf{\#}\textbf{\#}

\textbf{\#}\textbf{\#}\textbf{\#}\textbf{\#}\textbf{\#}\textbf{\#}\textbf{\#}
\textbf{\#}\textbf{\#}    \textbf{\#}
\textbf{\#}\textbf{\#}\textcolor{ansi-red-intense}{*} \textcolor{ansi-blue-intense}{@} \textbf{\#}
\textbf{\#}  \textcolor{ansi-cyan}{A}  \textbf{\#}
\textbf{\#}  \textbf{\#}\textcolor{ansi-magenta-intense}{o} \textbf{\#}
\textbf{\#}  \textbf{\#}\textbf{\#}\textbf{\#}\textbf{\#}
\textbf{\#}\textbf{\#}\textbf{\#}\textbf{\#}\textbf{\#}\textbf{\#}\textbf{\#}

\textbf{\#}\textbf{\#}\textbf{\#}\textbf{\#}\textbf{\#}\textbf{\#}\textbf{\#}
\textbf{\#}\textbf{\#}    \textbf{\#}
\textbf{\#}\textbf{\#}\textcolor{ansi-red-intense}{*}\textcolor{ansi-cyan}{A}\textcolor{ansi-blue-intense}{@} \textbf{\#}
\textbf{\#}     \textbf{\#}
\textbf{\#}  \textbf{\#}\textcolor{ansi-magenta-intense}{o} \textbf{\#}
\textbf{\#}  \textbf{\#}\textbf{\#}\textbf{\#}\textbf{\#}
\textbf{\#}\textbf{\#}\textbf{\#}\textbf{\#}\textbf{\#}\textbf{\#}\textbf{\#}

\textbf{\#}\textbf{\#}\textbf{\#}\textbf{\#}\textbf{\#}\textbf{\#}\textbf{\#}
\textbf{\#}\textbf{\#} \textcolor{ansi-cyan}{A}  \textbf{\#}
\textbf{\#}\textbf{\#}\textcolor{ansi-red-intense}{*} \textcolor{ansi-blue-intense}{@} \textbf{\#}
\textbf{\#}     \textbf{\#}
\textbf{\#}  \textbf{\#}\textcolor{ansi-magenta-intense}{o} \textbf{\#}
\textbf{\#}  \textbf{\#}\textbf{\#}\textbf{\#}\textbf{\#}
\textbf{\#}\textbf{\#}\textbf{\#}\textbf{\#}\textbf{\#}\textbf{\#}\textbf{\#}

\textbf{\#}\textbf{\#}\textbf{\#}\textbf{\#}\textbf{\#}\textbf{\#}\textbf{\#}
\textbf{\#}\textbf{\#}\textcolor{ansi-cyan}{A}   \textbf{\#}
\textbf{\#}\textbf{\#}\textcolor{ansi-red-intense}{*} \textcolor{ansi-blue-intense}{@} \textbf{\#}
\textbf{\#}     \textbf{\#}
\textbf{\#}  \textbf{\#}\textcolor{ansi-magenta-intense}{o} \textbf{\#}
\textbf{\#}  \textbf{\#}\textbf{\#}\textbf{\#}\textbf{\#}
\textbf{\#}\textbf{\#}\textbf{\#}\textbf{\#}\textbf{\#}\textbf{\#}\textbf{\#}

\textbf{\#}\textbf{\#}\textbf{\#}\textbf{\#}\textbf{\#}\textbf{\#}\textbf{\#}
\textbf{\#}\textbf{\#}    \textbf{\#}
\textbf{\#}\textbf{\#}\textcolor{ansi-cyan}{A} \textcolor{ansi-blue-intense}{@} \textbf{\#}
\textbf{\#} \textcolor{ansi-red-intense}{*}   \textbf{\#}
\textbf{\#}  \textbf{\#}\textcolor{ansi-magenta-intense}{o} \textbf{\#}
\textbf{\#}  \textbf{\#}\textbf{\#}\textbf{\#}\textbf{\#}
\textbf{\#}\textbf{\#}\textbf{\#}\textbf{\#}\textbf{\#}\textbf{\#}\textbf{\#}

\textbf{\#}\textbf{\#}\textbf{\#}\textbf{\#}\textbf{\#}\textbf{\#}\textbf{\#}
\textbf{\#}\textbf{\#}    \textbf{\#}
\textbf{\#}\textbf{\#}  \textcolor{ansi-blue-intense}{@} \textbf{\#}
\textbf{\#} \textcolor{ansi-cyan}{A}   \textbf{\#}
\textbf{\#} \textcolor{ansi-red-intense}{*}\textbf{\#}\textcolor{ansi-magenta-intense}{o} \textbf{\#}
\textbf{\#}  \textbf{\#}\textbf{\#}\textbf{\#}\textbf{\#}
\textbf{\#}\textbf{\#}\textbf{\#}\textbf{\#}\textbf{\#}\textbf{\#}\textbf{\#}

\textbf{\#}\textbf{\#}\textbf{\#}\textbf{\#}\textbf{\#}\textbf{\#}\textbf{\#}
\textbf{\#}\textbf{\#}    \textbf{\#}
\textbf{\#}\textbf{\#}  \textcolor{ansi-blue-intense}{@} \textbf{\#}
\textbf{\#}\textcolor{ansi-cyan}{A}    \textbf{\#}
\textbf{\#} \textcolor{ansi-red-intense}{*}\textbf{\#}\textcolor{ansi-magenta-intense}{o} \textbf{\#}
\textbf{\#}  \textbf{\#}\textbf{\#}\textbf{\#}\textbf{\#}
\textbf{\#}\textbf{\#}\textbf{\#}\textbf{\#}\textbf{\#}\textbf{\#}\textbf{\#}

\textbf{\#}\textbf{\#}\textbf{\#}\textbf{\#}\textbf{\#}\textbf{\#}\textbf{\#}
\textbf{\#}\textbf{\#}    \textbf{\#}
\textbf{\#}\textbf{\#}  \textcolor{ansi-blue-intense}{@} \textbf{\#}
\textbf{\#}     \textbf{\#}
\textbf{\#}\textcolor{ansi-cyan}{A}\textcolor{ansi-red-intense}{*}\textbf{\#}\textcolor{ansi-magenta-intense}{o} \textbf{\#}
\textbf{\#}  \textbf{\#}\textbf{\#}\textbf{\#}\textbf{\#}
\textbf{\#}\textbf{\#}\textbf{\#}\textbf{\#}\textbf{\#}\textbf{\#}\textbf{\#}

\textbf{\#}\textbf{\#}\textbf{\#}\textbf{\#}\textbf{\#}\textbf{\#}\textbf{\#}
\textbf{\#}\textbf{\#}    \textbf{\#}
\textbf{\#}\textbf{\#}  \textcolor{ansi-blue-intense}{@} \textbf{\#}
\textbf{\#}     \textbf{\#}
\textbf{\#} \textcolor{ansi-red-intense}{*}\textbf{\#}\textcolor{ansi-magenta-intense}{o} \textbf{\#}
\textbf{\#}\textcolor{ansi-cyan}{A} \textbf{\#}\textbf{\#}\textbf{\#}\textbf{\#}
\textbf{\#}\textbf{\#}\textbf{\#}\textbf{\#}\textbf{\#}\textbf{\#}\textbf{\#}

\textbf{\#}\textbf{\#}\textbf{\#}\textbf{\#}\textbf{\#}\textbf{\#}\textbf{\#}
\textbf{\#}\textbf{\#}    \textbf{\#}
\textbf{\#}\textbf{\#}  \textcolor{ansi-blue-intense}{@} \textbf{\#}
\textbf{\#}     \textbf{\#}
\textbf{\#} \textcolor{ansi-red-intense}{*}\textbf{\#}\textcolor{ansi-magenta-intense}{o} \textbf{\#}
\textbf{\#} \textcolor{ansi-cyan}{A}\textbf{\#}\textbf{\#}\textbf{\#}\textbf{\#}
\textbf{\#}\textbf{\#}\textbf{\#}\textbf{\#}\textbf{\#}\textbf{\#}\textbf{\#}

\textbf{\#}\textbf{\#}\textbf{\#}\textbf{\#}\textbf{\#}\textbf{\#}\textbf{\#}
\textbf{\#}\textbf{\#}    \textbf{\#}
\textbf{\#}\textbf{\#}  \textcolor{ansi-blue-intense}{@} \textbf{\#}
\textbf{\#} \textcolor{ansi-red-intense}{*}   \textbf{\#}
\textbf{\#} \textcolor{ansi-cyan}{A}\textbf{\#}\textcolor{ansi-magenta-intense}{o} \textbf{\#}
\textbf{\#}  \textbf{\#}\textbf{\#}\textbf{\#}\textbf{\#}
\textbf{\#}\textbf{\#}\textbf{\#}\textbf{\#}\textbf{\#}\textbf{\#}\textbf{\#}

\textbf{\#}\textbf{\#}\textbf{\#}\textbf{\#}\textbf{\#}\textbf{\#}\textbf{\#}
\textbf{\#}\textbf{\#}    \textbf{\#}
\textbf{\#}\textbf{\#}  \textcolor{ansi-blue-intense}{@} \textbf{\#}
\textbf{\#} \textcolor{ansi-red-intense}{*}   \textbf{\#}
\textbf{\#}\textcolor{ansi-cyan}{A} \textbf{\#}\textcolor{ansi-magenta-intense}{o} \textbf{\#}
\textbf{\#}  \textbf{\#}\textbf{\#}\textbf{\#}\textbf{\#}
\textbf{\#}\textbf{\#}\textbf{\#}\textbf{\#}\textbf{\#}\textbf{\#}\textbf{\#}

\textbf{\#}\textbf{\#}\textbf{\#}\textbf{\#}\textbf{\#}\textbf{\#}\textbf{\#}
\textbf{\#}\textbf{\#}    \textbf{\#}
\textbf{\#}\textbf{\#}  \textcolor{ansi-blue-intense}{@} \textbf{\#}
\textbf{\#}\textcolor{ansi-cyan}{A}\textcolor{ansi-red-intense}{*}   \textbf{\#}
\textbf{\#}  \textbf{\#}\textcolor{ansi-magenta-intense}{o} \textbf{\#}
\textbf{\#}  \textbf{\#}\textbf{\#}\textbf{\#}\textbf{\#}
\textbf{\#}\textbf{\#}\textbf{\#}\textbf{\#}\textbf{\#}\textbf{\#}\textbf{\#}

\textbf{\#}\textbf{\#}\textbf{\#}\textbf{\#}\textbf{\#}\textbf{\#}\textbf{\#}
\textbf{\#}\textbf{\#}    \textbf{\#}
\textbf{\#}\textbf{\#}  \textcolor{ansi-blue-intense}{@} \textbf{\#}
\textbf{\#} \textcolor{ansi-cyan}{A}\textcolor{ansi-red-intense}{*}  \textbf{\#}
\textbf{\#}  \textbf{\#}\textcolor{ansi-magenta-intense}{o} \textbf{\#}
\textbf{\#}  \textbf{\#}\textbf{\#}\textbf{\#}\textbf{\#}
\textbf{\#}\textbf{\#}\textbf{\#}\textbf{\#}\textbf{\#}\textbf{\#}\textbf{\#}

\textbf{\#}\textbf{\#}\textbf{\#}\textbf{\#}\textbf{\#}\textbf{\#}\textbf{\#}
\textbf{\#}\textbf{\#}    \textbf{\#}
\textbf{\#}\textbf{\#}  \textcolor{ansi-blue-intense}{@} \textbf{\#}
\textbf{\#}  \textcolor{ansi-cyan}{A}\textcolor{ansi-red-intense}{*} \textbf{\#}
\textbf{\#}  \textbf{\#}\textcolor{ansi-magenta-intense}{o} \textbf{\#}
\textbf{\#}  \textbf{\#}\textbf{\#}\textbf{\#}\textbf{\#}
\textbf{\#}\textbf{\#}\textbf{\#}\textbf{\#}\textbf{\#}\textbf{\#}\textbf{\#}

\textbf{\#}\textbf{\#}\textbf{\#}\textbf{\#}\textbf{\#}\textbf{\#}\textbf{\#}
\textbf{\#}\textbf{\#}    \textbf{\#}
\textbf{\#}\textbf{\#} \textcolor{ansi-cyan}{A}\textcolor{ansi-blue-intense}{@} \textbf{\#}
\textbf{\#}   \textcolor{ansi-red-intense}{*} \textbf{\#}
\textbf{\#}  \textbf{\#}\textcolor{ansi-magenta-intense}{o} \textbf{\#}
\textbf{\#}  \textbf{\#}\textbf{\#}\textbf{\#}\textbf{\#}
\textbf{\#}\textbf{\#}\textbf{\#}\textbf{\#}\textbf{\#}\textbf{\#}\textbf{\#}

\textbf{\#}\textbf{\#}\textbf{\#}\textbf{\#}\textbf{\#}\textbf{\#}\textbf{\#}
\textbf{\#}\textbf{\#} \textcolor{ansi-cyan}{A}  \textbf{\#}
\textbf{\#}\textbf{\#}  \textcolor{ansi-blue-intense}{@} \textbf{\#}
\textbf{\#}   \textcolor{ansi-red-intense}{*} \textbf{\#}
\textbf{\#}  \textbf{\#}\textcolor{ansi-magenta-intense}{o} \textbf{\#}
\textbf{\#}  \textbf{\#}\textbf{\#}\textbf{\#}\textbf{\#}
\textbf{\#}\textbf{\#}\textbf{\#}\textbf{\#}\textbf{\#}\textbf{\#}\textbf{\#}

\textbf{\#}\textbf{\#}\textbf{\#}\textbf{\#}\textbf{\#}\textbf{\#}\textbf{\#}
\textbf{\#}\textbf{\#}  \textcolor{ansi-cyan}{A} \textbf{\#}
\textbf{\#}\textbf{\#}  \textcolor{ansi-blue-intense}{@} \textbf{\#}
\textbf{\#}   \textcolor{ansi-red-intense}{*} \textbf{\#}
\textbf{\#}  \textbf{\#}\textcolor{ansi-magenta-intense}{o} \textbf{\#}
\textbf{\#}  \textbf{\#}\textbf{\#}\textbf{\#}\textbf{\#}
\textbf{\#}\textbf{\#}\textbf{\#}\textbf{\#}\textbf{\#}\textbf{\#}\textbf{\#}

\textbf{\#}\textbf{\#}\textbf{\#}\textbf{\#}\textbf{\#}\textbf{\#}\textbf{\#}
\textbf{\#}\textbf{\#}   \textcolor{ansi-cyan}{A}\textbf{\#}
\textbf{\#}\textbf{\#}  \textcolor{ansi-blue-intense}{@} \textbf{\#}
\textbf{\#}   \textcolor{ansi-red-intense}{*} \textbf{\#}
\textbf{\#}  \textbf{\#}\textcolor{ansi-magenta-intense}{o} \textbf{\#}
\textbf{\#}  \textbf{\#}\textbf{\#}\textbf{\#}\textbf{\#}
\textbf{\#}\textbf{\#}\textbf{\#}\textbf{\#}\textbf{\#}\textbf{\#}\textbf{\#}

\textbf{\#}\textbf{\#}\textbf{\#}\textbf{\#}\textbf{\#}\textbf{\#}\textbf{\#}
\textbf{\#}\textbf{\#}    \textbf{\#}
\textbf{\#}\textbf{\#}  \textcolor{ansi-blue-intense}{@}\textcolor{ansi-cyan}{A}\textbf{\#}
\textbf{\#}   \textcolor{ansi-red-intense}{*} \textbf{\#}
\textbf{\#}  \textbf{\#}\textcolor{ansi-magenta-intense}{o} \textbf{\#}
\textbf{\#}  \textbf{\#}\textbf{\#}\textbf{\#}\textbf{\#}
\textbf{\#}\textbf{\#}\textbf{\#}\textbf{\#}\textbf{\#}\textbf{\#}\textbf{\#}

\textbf{\#}\textbf{\#}\textbf{\#}\textbf{\#}\textbf{\#}\textbf{\#}\textbf{\#}
\textbf{\#}\textbf{\#}    \textbf{\#}
\textbf{\#}\textbf{\#} \textcolor{ansi-red-intense}{*}\textcolor{ansi-cyan}{A} \textbf{\#}
\textbf{\#}   \textcolor{ansi-red-intense}{*} \textbf{\#}
\textbf{\#}  \textbf{\#}\textcolor{ansi-magenta-intense}{o} \textbf{\#}
\textbf{\#}  \textbf{\#}\textbf{\#}\textbf{\#}\textbf{\#}
\textbf{\#}\textbf{\#}\textbf{\#}\textbf{\#}\textbf{\#}\textbf{\#}\textbf{\#}

\textbf{\#}\textbf{\#}\textbf{\#}\textbf{\#}\textbf{\#}\textbf{\#}\textbf{\#}
\textbf{\#}\textbf{\#}    \textbf{\#}
\textbf{\#}\textbf{\#} \textcolor{ansi-red-intense}{*}\textcolor{ansi-magenta-intense}{o} \textbf{\#}
\textbf{\#}   \textcolor{ansi-cyan}{A} \textbf{\#}
\textbf{\#}  \textbf{\#}\textcolor{ansi-blue-intense}{@} \textbf{\#}
\textbf{\#}  \textbf{\#}\textbf{\#}\textbf{\#}\textbf{\#}
\textbf{\#}\textbf{\#}\textbf{\#}\textbf{\#}\textbf{\#}\textbf{\#}\textbf{\#}

\textbf{\#}\textbf{\#}\textbf{\#}\textbf{\#}\textbf{\#}\textbf{\#}\textbf{\#}
\textbf{\#}\textbf{\#}    \textbf{\#}
\textbf{\#}\textbf{\#} \textcolor{ansi-red-intense}{*}\textcolor{ansi-magenta-intense}{o} \textbf{\#}
\textbf{\#}  \textcolor{ansi-cyan}{A}  \textbf{\#}
\textbf{\#}  \textbf{\#}\textcolor{ansi-blue-intense}{@} \textbf{\#}
\textbf{\#}  \textbf{\#}\textbf{\#}\textbf{\#}\textbf{\#}
\textbf{\#}\textbf{\#}\textbf{\#}\textbf{\#}\textbf{\#}\textbf{\#}\textbf{\#}

\textbf{\#}\textbf{\#}\textbf{\#}\textbf{\#}\textbf{\#}\textbf{\#}\textbf{\#}
\textbf{\#}\textbf{\#}    \textbf{\#}
\textbf{\#}\textbf{\#} \textcolor{ansi-red-intense}{*}\textcolor{ansi-magenta-intense}{o} \textbf{\#}
\textbf{\#} \textcolor{ansi-cyan}{A}   \textbf{\#}
\textbf{\#}  \textbf{\#}\textcolor{ansi-blue-intense}{@} \textbf{\#}
\textbf{\#}  \textbf{\#}\textbf{\#}\textbf{\#}\textbf{\#}
\textbf{\#}\textbf{\#}\textbf{\#}\textbf{\#}\textbf{\#}\textbf{\#}\textbf{\#}

\textbf{\#}\textbf{\#}\textbf{\#}\textbf{\#}\textbf{\#}\textbf{\#}\textbf{\#}
\textbf{\#}\textbf{\#}    \textbf{\#}
\textbf{\#}\textbf{\#}\textcolor{ansi-cyan}{A}\textcolor{ansi-red-intense}{*}\textcolor{ansi-magenta-intense}{o} \textbf{\#}
\textbf{\#}     \textbf{\#}
\textbf{\#}  \textbf{\#}\textcolor{ansi-blue-intense}{@} \textbf{\#}
\textbf{\#}  \textbf{\#}\textbf{\#}\textbf{\#}\textbf{\#}
\textbf{\#}\textbf{\#}\textbf{\#}\textbf{\#}\textbf{\#}\textbf{\#}\textbf{\#}

\textbf{\#}\textbf{\#}\textbf{\#}\textbf{\#}\textbf{\#}\textbf{\#}\textbf{\#}
\textbf{\#}\textbf{\#}    \textbf{\#}
\textbf{\#}\textbf{\#} \textcolor{ansi-cyan}{A}\textcolor{ansi-blue-intense}{@} \textbf{\#}
\textbf{\#}     \textbf{\#}
\textbf{\#}  \textbf{\#}\textcolor{ansi-blue-intense}{@} \textbf{\#}
\textbf{\#}  \textbf{\#}\textbf{\#}\textbf{\#}\textbf{\#}
\textbf{\#}\textbf{\#}\textbf{\#}\textbf{\#}\textbf{\#}\textbf{\#}\textbf{\#}

Número de passos: 43
Números de moves: 31
Números de pushes: 12

    \end{Verbatim}


    % Add a bibliography block to the postdoc
    
    
    
    \end{document}
